\documentclass[8pt,onesided]{article}

% Packages.

\usepackage[usenames,dvipsnames]{xcolor}
\usepackage{amsmath}
\usepackage{mathtools}
\usepackage{amssymb}
\usepackage{setspace}
\usepackage{graphicx}
\usepackage[colorlinks=true,linktoc=all,linkcolor=blue,urlcolor=blue,citecolor=red]{hyperref}
\usepackage{amscd,amsthm,curve2e,graphicx,manfnt,mathdots,mathrsfs}
\usepackage{pgfplots}
\usepackage{graphics}
\usepackage{caption}
\usepackage{geometry}
\usepackage{fancyhdr}
\usepackage{titlesec}
\usepackage{newtxtext}
\usepackage{newtxmath}
\usepackage[ruled,linesnumbered]{algorithm2e}
\usepackage{float}
\usepackage{paralist}   
\usepackage{pstricks-add}
\usepackage{tikz-cd}
\usepackage{pst-pdf}
\usepackage{extarrows}

\usepackage{physics}
% User Commands.

\newcommand{\rnd}{\partial}
\newcommand{\dif}{\mathrm{d}\.}

\newcommand{\dq}[2]{\frac{\mathrm{d} #1 }{\mathrm{d} #2}}
\newcommand{\hdq}[3]{\frac{\mathrm{d}^{#3} #1 }{\mathrm{d} #2}^{#3}}
\newcommand{\pdq}[2]{\frac{\partial #1}{\partial #2}}
\newcommand{\hpdq}[3]{\frac{\partial^{#3} #1}{\partial #2 ^{#3}}}

\newcommand{\iprod}[1]{\left\langle #1\right\rangle}
\newcommand{\nm}[1]{\left\lVert#1\right\rVert}

\newcommand{\ex}[1]{\mathbb{E}\left[#1\right]}

\newcommand{\re}{\mathbb{R}}
\newcommand{\ra}{\mathbb{Q}}
\newcommand{\zah}{\mathbb{Z}}
\newcommand{\neu}{\mathbb{N}}
\newcommand{\s}{\mathbb{S}}
\newcommand{\p}{\mathbb{P}}
\newcommand{\f}{\mathscr{F}}
\newcommand{\B}{\mathscr{B}}
\newcommand{\w}{\mathbb{W}}
\newcommand{\q}{\mathbb{Q}}
\newcommand{\e}{\mathbb{E}}

\newcommand{\mc}{\color{BlueViolet}}
\renewcommand{\qedsymbol}{\mc $\blacksquare$}

\renewcommand{\ge}{\geqslant}
\renewcommand{\le}{\leqslant}

\definecolor{MyRed}{RGB}{117,35,31}
\newcommand{\mr}{\color{MyRed}}
\usepackage[colorlinks=true,linktoc=all,linkcolor=MyRed,urlcolor=blue,citecolor=MyRed]{hyperref}

% Theorems, etc. Environment Settings.

\theoremstyle{definition}
\newtheorem{problem}{\mc Problem}
\newtheorem*{proposition}{\mc Proposition}
\newtheorem{definition}{\mc Definition}
\newtheorem{theorem}{\mc Theorem}
\newtheorem*{example}{\mc Example}
\newtheorem*{remark}{\mc Remark}
\newtheorem{lemma}{\mc Lemma}
\newtheorem*{ass}{\mc Assumption}
\newtheorem{coro}{\mc Corollary}


\newenvironment{myenume}{\begin{enumerate}[(i).]\setlength{\itemsep}{0pt}\setlength{\itemindent}{1em}}{\end{enumerate}}

\bibliographystyle{apalike}

\globalcolorstrue

\titleformat{\section}{\color{BlueViolet}\normalfont\large\bf}{\color{BlueViolet}\S\thesection}{1em}{}
\titleformat{\subsection}{\color{BlueViolet}\normalfont\bf}{\color{BlueViolet}\thesubsection}{1em}{}

\geometry{left=2.0cm, right=2.0cm, top=2.5cm, bottom=2.5cm}

\fancypagestyle{style}{
    \chead{Final Essay of Martingale Theory and Stochastic Integration}
    \renewcommand{\headrulewidth}{.5pt}  
    \lhead{}
    \rhead{}
    \fancyfoot[L]{\color{gray}Beining Wu - Fall 2021}
    \fancyfoot[R]{\thepage}
    \fancyfoot[C]{}
    }
\pagestyle{style}

\newcommand{\hl}[1]{\textbf{\color{MyRed}#1}}


\begin{document}

{
\title{\color{BlueViolet} On the Uniqueness and Existence of Stochastic Differential Equations with Infinite Delay and L\'evy Noise\vspace*{.5em}\\  \large Final Essay of Martingale Theory and Stochastic Integration}
\author{Beining Wu\footnote{mail: \texttt{andrewwu@mail.ustc.edu.cn}, Student ID: PB19151833}}
\maketitle
}

\begin{abstract}
    In this essay we are going to study one type of \textbf{Stochastic Differential Delay Equation (SDDE)}, driven with Brownian motion and Poisson random measure and polynomial growth. In section \ref{ss2}, we will give our formal definition of the uniqueness and existence in sense of this special equation. In section \ref{ss3} we will introduce some assumptions and the main result of our essay. In section \ref{ss4} we use the assumptions before to give a formal proof of the main theorem.
\end{abstract}

\section{Introduction, Problem Setting}

The equation we are going to study can be formulated as
\begin{equation}
    \label{sdde}
    \dd X_t=f(X_t)+\int_{ -\infty}^0 X_{t+r} g(r) \dd r \dd t+\sigma(t,X_t)\dd B_t+\int_Z b(t, X_{t-},z)\tilde N_p(\dd z \dd t).
\end{equation}

This equation differs from the classical SDE which we have concerned before, because of involving the delay term and Poisson random noise. Specifically, the equation \ref{sdde} involves the integral memory term which preserve the whole past information. Consequently this results in more difficult analysis.

Here we present with full problem setting of the equation \ref{sdde}. Assume that we have fixed filtered space $(\Omega, \f, (\f_t)_{t\ge 0},\p)$. Let $B=(B_t)_{t\ge 0}$ be the standard Brownian motion on the space defined above. Let the measure space $(Z, \B, \nu)$ be the base of our adapted stationary Poisson point process $p$ with induced Poisson random measure $N_p$. Denote the compensated Poisson process as
\begin{equation*}
    \tilde N_p:=\hat N_p(\dd z \dd t)- \dd t \nu (\dd z)
\end{equation*}

Moreover, the equation \ref{sdde} is given with the initial data $X_0=\xi$, by this we actually means $\xi=\xi_t \in L^2((-\infty,0], \re)$, a deterministic function which is actually the path before $t=0$. 

Additionally, we assume that $\sigma$ and $b$ are global Lipschitz function, while $f,g:\re \to \re$ are non-random function. And $f$ is assumed to be a polynomial.

\paragraph{Problem} On what additional conditions can we deduce the uniqueness and existence result of the equation \ref{sdde}?

We're going to give a possible solution of this problem in the following sections. 

\section{Definition of Uniqueness and Existence}
\label{ss2}

To begin with, we shall clarify the definition of existence and uniqueness for this special equation.

\begin{definition}
[Pathwise Uniqueness] For fixed set-up $(\Omega,\f, (\f_t )_{t\ge 0},\p,B,N)$, if there are two semimartingales $Y$ and $Y'$ that solves the equation \ref{sdde}, then they are indistinguishable, i.e., a.s. with same trajectory.
\end{definition}

Furthermore, since the equation differs from the usual Brownian-motion driven equation with jumps, we shall introduce some additional definitions of solutions.

\begin{definition}
[Global and Local Solution] For fixed set-up $(\Omega,\f, (\f_t )_{t\ge 0},\p,B,N)$, we say the $\f_t$-adapted semimartingale $X$ is the \textbf{global solution} of the equation \ref{sdde} if we have
\begin{enumerate}
    \item $X_t=\xi_t,\,\forall t\le 0$.
    \item $\p$-a.s. $X_t$ is càdlàg.
    \item $\forall t\ge 0$ we have
%remains
\begin{equation*}
    \p\text{-a.s.} \quad \left\{
    \begin{aligned}
        &\int_0^t |\sigma(s,X_s)|^2 \dd s<\infty\\
       & \int_0^t |f(X_s)| \dd s<\infty\\
       & \int_0^t \int_Z |b(s,X_s,z)|^2 \nu(\dd z)\dd s<\infty\\
    \end{aligned}
    \right.
\end{equation*}
    \item $\p$-a.s., the following equation is true.
\end{enumerate}
\begin{equation*}
    X_t= \xi_0+\int_0^t f(X_s)\dd s +\int_0^t \int_{-\infty}^0 X_{s+r} g(r)\dd r \dd s +\int_0^t \sigma(s, X_s)\dd B_s +\int_0^t \int_Z b(s, X_{s-},z)\tilde N_p(\dd z \dd s), \quad \forall t\ge0.
\end{equation*}

Moreover, if this equation is true only before the stopping time $\tau$, i.e.,
\begin{equation*}
    X_{t\wedge\tau}= \xi_0+\int_0^{t\wedge\tau} f(X_s)\dd s +\int_0^{t\wedge\tau} \int_{-\infty}^0 X_{s+r} g(r)\dd r \dd s +\int_0^{t\wedge\tau} \sigma(s, X_s)\dd B_s +\int_0^{t\wedge\tau} \int_Z b(s, X_{s-},z)\tilde N_p(\dd z \dd s), \quad \forall  t\ge0.
\end{equation*}
Then we say that the couple $(X,\tau )$ is said to be the local solution of the equation \ref{sdde}. If $\tau$ is the explosion time of $X$ and there is a increasing sequence of stopping time $\tau_n\nearrow \tau$, and $(X,\tau_n)$ are all the local solutions, then $(X,\tau)$ is said to be the maximal local solution.
\end{definition}


\section{Main Result}
\label{ss3}

\begin{ass}
    The following assumptions may be critical in our proof.
\begin{enumerate}[{A}.1]
    \item (Global Lipschitz of $\sigma$ and $b$) In the following context, we always regard $\sigma$ and $b$ to be globally Lipschitz continuous, i.e.
\begin{equation*}
|\sigma(s,x)-\sigma(s,y)|\wedge |b(s,x,z)-b(s,y,z)|\le L|x-y|, \quad\forall x,y\in \re, z\in Z, s\in \re.
\end{equation*}
     \item (Linear Growth) $f$ is a linear function, $f(x)=a_0+a_1x$, and without loss of generality $|a_1|\le L$. This implies that $f$ is also $L$ Lipschitz function.
     \item (Nonlinear Polynomial Growth) The order of $f$ is no less than $2$, this means $f$ is locally Lipschitz.
     \item (Integrability) We assume $g\in L^2((-\infty,0],\re)$.
\end{enumerate}
\end{ass}

Based on the aforementioned assumptions, we present our main results in this essay.

\begin{theorem}[Global]
Under assumptions A.1, A.2 and A.4, the equation \ref{sdde} exists a unique global solution(modulo indistinguishable).
\end{theorem}

\begin{theorem}
[Local] Under assumptions A.1, A.3 and A.4, the equation \ref{sdde} exists a unique maximal local solution.
\end{theorem}


\section{Proof of Main Result}
\label{ss4}


\subsection{Uniqueness}

The first part concerns with the uniqueness. Let $X$ and $Y$ be the $\tau$-localized solution, on the finite interval $[0,T]$. Moreover, define the stopping time
\begin{equation*}
    \tau_n =\tau \wedge \{s: |X_s|\vee|Y_s|\ge n \}.
\end{equation*}
Define $X^n_t=X_{t\wedge \tau_n}$ and $Y^n_t=Y_{t\wedge \tau_n}$. Consider the square expectation of the difference
\begin{align*}
    \e\big[(X^n_t -Y^n_t)^2\big]=& \e  \Bigg[\Big(
    \int_0^t \big(f(X_s^Nn)-f(Y_s^n)\big)\dd s
    +\int_0^t \int_{-\infty}^0 \big(X_{s+r}^n-Y_{s+r}^n\big) g(r)\dd r \dd s
 +\int_0^t \big(\sigma(s, X^n_s)-\sigma(s, Y_s^n)\big)\dd B_s \\
 &+\int_0^t \int_Z \big(b(s, X^n_{s-},z)-b(s, Y^n_{s-},z)\big)\tilde N_p(\dd z \dd s)\Big)^2\Bigg]
\end{align*}
By the basic arithmetic-square mean value inequality, 
\begin{align*}
    \text{RHS}\le& 4 \e \Big[\Big(\int_0^t \big(f(X_s^n)-f(Y_s^n)\big)\dd s\Big)^2\Big]+4\e \Big[\Big(\int_0^t \int_{-\infty}^0 \big(X_{s+r}^n-Y_{s+r}^n\big) g(r)\dd r \dd s\Big)^2\Big]\\
    &+4\e\Big[\Big(\int_0^t \big(\sigma(s, X^n_s)-\sigma(s, Y_s^n)\big)\dd B_s \Big)^2\Big]+4\e\Big[\Big(\int_0^t \int_Z \big(b(s, X^n_{s-},z)-b(s, Y^n_{s-},z)\big)\tilde N_p(\dd z \dd s)\Big)^2\Big].
\end{align*}
We bound it by term. Under assumption A.1, the linear growth and global Lipschitz property is true and therefore the first term can be written as
\begin{align*}
    4 \e \Big[\Big(\int_0^t \big(f(X_s^n)-f(Y_s^n)\big)\dd s\Big)^2\Big]&\le 4\e\Big[ t\int_0^t\big(f(X_s^N)-f(Y_s^n)\big)^2\dd s \Big]\\
    &\le 4TL^2 \e\Big[ \int_0^t  (X^n_s-Y^n_s)^2\dd s\Big].
\end{align*}

Now turn to the second term, which involves the past memory. Since both $X,Y$ solve the equation, they have identical past data before $t=0$. So
\begin{align*}
    4\e \Big[\Big(\int_0^t \int_{-\infty}^0 \big(X_{s+r}^n-Y_{s+r}^n\big) g(r)\dd r \dd s\Big)^2\Big]
    =&4\e \Big[\Big(\int_0^t \int_{-s}^0 \big(X_{s+r}^n-Y_{s+r}^n\big) g(r)\dd r \dd s\Big)^2\Big]\\
    \le&4\e \Big[\Big(\int_0^t \int_{0}^s \big(X_{r}^n-Y_{r}^n\big) g(r-s)\dd r \dd s\Big)^2\Big]\\
    \le&4\e \Bigg[\Bigg(\int_0^t \Big(\int_{0}^s \big(X_{r}^n-Y_{r}^n\big)^2\dd r\Big)^{1/2}\Big(\int_{0}^s g(r-s)^2\dd r\Big)^{1/2} \dd s\Bigg)^2\Bigg]\\
    \le&4\e \Bigg[\Bigg(\int_0^t \Big(\int_{0}^s \big(X_{r}^n-Y_{r}^n\big)^2\dd r\Big)^{1/2}\nm{g}_{L^2((-\infty,0])} \dd s\Bigg)^2\Bigg]\\
    \le &4\nm{g}_{L^2((-\infty,0])}^2  T^2 \e  \Big[\int_0^t \big(X_{r}^n-Y_{r}^n\big)^2 \dd r\Big].
\end{align*}

For the third drift term, global Lipschitz continuity and standard Ito's isometry gives 
\begin{align*}
    4\e\Big[\Big(\int_0^t \big(\sigma(s, X^n_s)-\sigma(s, Y_s^n)\big)\dd B_s \Big)^2\Big]\le &
    4\e\Big[ \int_0^t  \Big(\sigma(s,X^n_s)-\sigma(s,Y^n_s) \Big)^2 \dd s\Big]\\
    \le& 4 \e \Big[ \int_0^t L^2 \big(X^n_s-Y^n_s\big)^2 \dd s \Big].
\end{align*}

And then comes to the jump term, 
\begin{align*}
    4\e\Big[\Big(\int_0^t \int_Z \big(b(s, X^n_{s-},z)-b(s, Y^n_{s-},z)\big)\tilde N_p(\dd z \dd s)\Big)^2\Big]\le&4L^2\e\Big[\int_0^t  \int_Z   \big(X^n_s-Y^n_s\big)^2 \nu(\dd z)\dd s  \Big]\\
    \le& 4L^2\nu(Z) \e\Big[\int_Z   \big(X^n_s-Y^n_s\big)^2 \nu(\dd z)\dd s\Big].
\end{align*}

Let $h(t)=\e[(X^n_t-Y^n_t)^2]$, then combine the last four bounds to obtain
\begin{equation*}
    h(t)\le C_{T,g,L} \int_0^t h(s)\dd s, \quad t\le T.
\end{equation*}
By the Grownall's inequality, we now have
\begin{equation*}
    h(t)=0,\quad \forall t\le T.
\end{equation*}
Thus, any two solutions of the equation \ref{sdde} (could be both global or local) must be a.s. indistinguishable, thus we're done.

\subsection{Global Existence}

The second part concerns with the existence. It's based on the standard Picard approximation. Consider the class
\begin{equation*}
    H_T=\{h: \;h\text{ is }\f_t \text{-adapted and }\p\text{-a.s. càdlàg, with }h(0)=\xi\}.
\end{equation*}
And define the norm on $H_T$ as
\begin{equation*}
\nm{h}:=\e\big[ \sup_{s\le T} |h_s|^2\big].
\end{equation*}

Consider the linear operator $J:H_T\to H_T$,
\begin{equation*}
    h\mapsto Jh:=(Jh)_t=\xi_0+\int_0^t f(h_s)\dd s +\int_0^t \int_{-\infty}^0 h_{s+r} g(r)\dd r \dd s +\int_0^t \sigma(s, h_s)\dd B_s +\int_0^t \int_Z b(s, h_{s-},z)\tilde N_p(\dd z \dd s), \quad \forall T\ge t\ge0.
\end{equation*}

Then, if $\nm{h}_{H_T}<\infty$, then $\nm{Jh}_{H_T}<\infty$.

%Remains

On the other hand, if $h,h'\in H_T$, then we shall bound $\nm{Jh-Jh'}$ in the following sense.

\begin{align*}
\e\big[ \sup_{t\le T}(Jh_t-Jh'_t)^2\big]\le 4 \e\Bigg[ 
\sup_{t\le T}\Big(\int_0^t \big(f(h_s)-f(h'_s)\big)\dd s \Big)^2
+\sup_{t\le T} \Big(\int_0^t\int_{-\infty}^0 \big(h_{s+r}-h'_{s+r}\big) g(r)\dd r \dd s\Big)^2\\
+\sup_{t\le T} \Big( \int_0^t \big(\sigma(s,h_s)-\sigma(s,h'_s)\big)\dd B_s\Big)
+ \sup_{t\le T}\Big(\int_0^t \int_Z \big(b(s, h_{s-},z)-b(s, h'_{s-},z)\big)\tilde N_p(\dd z \dd s)\Big)^2
\Bigg].
\end{align*}
 Again we bound each term separately. For the first term
\begin{align*}
    4 \e\Bigg[ 
        \sup_{t\le T}\Big(\int_0^t \big(f(h_s)-f(h'_s)\big)\dd s \Big)^2\Bigg]\le4T\e \Bigg[\int_0^T \big(f(h_s)-f(h'_s)\big)^2\dd s\Bigg]\\
        \le 4T \e\Biggl[ \int_0^T L^2 \sup_{u\le s} \big(h'_u-h_u\big)^2 \dd s \Biggr].
\end{align*}
For the memory term.
\begin{align*}
4\e\Biggl[ \sup_{t\le T }\Biggl(  \int_0^t \int_{-\infty}^0 \bigl( h_{r+s} - h'_{r+s} \bigr)g(r)\dd r \dd s\Biggr)^2\Biggr]
&\le
4\e\Biggl[ \sup_{t\le T }\Biggl(  \int_0^t \int_{0}^s \bigl( h_{r} - h'_{r} \bigr)g(r-s)\dd r \dd s\Biggr)^2\Biggr]\\
&\le4\e\Biggl[ \sup_{t\le T} \Biggl(   \int_0^t \bigl(\int_0^s (h_r - h'_r)^2\dd r   \bigr)^{1/2}\bigl(\int_0^s g^2(r-s)\dd r   \bigr)^{1/2} \dd s\Biggr)^2\Biggr]\\
&\le4 \nm{g}_{L^2((-\infty,0])} \e \Biggl[ \sup_{t\le T} \int_0^t \biggl( \int_0^s  \big(h_r-h'_r\big)^2 \dd r \biggr)\dd s \Biggr]\\
&\le 4\nm{g}_{L^2((-\infty,0])} \e \Biggl[ \int_0^T  T \sup_{r\le s}\big(
    h_r -h'_r\big)^2\dd s \Biggr].
\end{align*}
For the drift term, we shall use the \hyperref[bdg]{Burkholder-Davis-Gundy inequality} to bound
\begin{align*}
    4\e \Biggl[\sup_{t\le T}\Big( \int_0^t \big(\sigma(s,h_s)-\sigma(s,h'_s)\big)\dd B_s\Big)^2\Biggr]\le& 4C_p\e \Biggl[\int_0^T \big(\sigma(s,h_s)-\sigma(s,h'_s)\big)^2 \dd s\Biggr]\\
    \le& 4C_pL^2 \e\Biggl[ \int_0^T \sup_{u\le s}(h'_u-h_u)^2 \dd s \Biggr].
\end{align*}
And finally to the jump term, by the \hyperref[bdgg]{generalized Burkholder-Davis-Gundy inequality}, we have
\begin{align*}
    4\e\Biggl[ \sup_{t\le T}\Big(\int_0^t \int_Z \big(b(s, h_{s-},z)-b(s, h'_{s-},z)\big)\tilde N_p(\dd z \dd s)\Big)^2 \Biggr]
    \le
    4C'_p \e\Biggl[ \int_0^{T+}\int_Z \big(b(s, h_{s-},z)-b(s, h'_{s-},z) \big)^2N_p(\dd z\dd s)\Biggr]\\
    \le
     4C'_p\Biggl[ \int_0^T \int_ZL^2\big(h_s-h'_s\big)^2 \nu(\dd z)\dd s\Biggr]
     \le
     4C'_p L^2\nu(Z)\e\biggl[\int_0^T\big(h_s-h'_s\big)^2\dd s \biggr].
\end{align*}
Combine all the bounds above to get
\begin{equation}
    \label{diff}
\e\biggl[ \sup_{t\le T} (Jh_t-Jh'_t)^2\biggr]\le C_{p,T,Z,L}\e\biggl[\int_0^T \sup_{u\le t}(h_u-h'_u)^2 \dd t\biggr].
\end{equation}

Now we construct a sequence, $h^n=Jh^{n-1}$, define
\begin{equation*}
    g_n(t)=\e\biggl[\sup_{s\le t}(h^n_s-h^{n-1}_s)^2\biggr].
\end{equation*}
By the estimation \ref{diff}, we have
\begin{equation*}
    g_n(t)\le C\int_0^t g_{n-1}(s)\dd s.
\end{equation*}
This inequality implies that
\begin{equation*}
    g_n(t)\le C_0 C^{n-1} \frac{t^{n-1}}{(n-1)!}.
\end{equation*}
In particular, we have $\sum_n g_n(T)^{1/2}<\infty$, which implies that
\begin{equation*}
    \p\text{-a.s.}\quad \sum_{n} \sup_{s\le T}|h_s^n-h_s^{n-1}| <\infty.
\end{equation*}
This means, $h^n$ converges uniformly to a limit $h$. \hyperref[ucd]{Theorem 3} implies that the limit process is (a.s.) càdlàg. And by a limit process we can prove that the limit is exactly the solution.

\subsection{Localization}

The following part is based on assumption A.1, A.3 and A.4, therefore the $f$ is only local Lipschitz. We shall prove the local result with a truncation technique.

Define the truncated growth function
\begin{equation*}
    f_N(x)=f(\frac{|x|\wedge N}{|x|}x).
\end{equation*}
Since $f$ is a polynomial, $f_N$ is globally Lipschitz continuous with constant $L_N$. So, the global scenario implies that for fixed set-up, following equation
\begin{equation*}
    \dd X_t=f_N(X_t)+\int_{ -\infty}^0 X_{t+r} g(r) \dd r \dd t+\sigma(t,X_t)\dd B_t+\int_Z b(t, X_{t-},z)\tilde N_p(\dd z \dd t).
\end{equation*}
has unique global solution (up to indistinguishable). Note that before the stopping time $\tau_N:=\inf\{t:|X_t|\ge N\}$, the process $X_{t\wedge \tau_N}$ solves the equation \ref{sdde}. Hence $(X,\tau_N)$ is a local solution of equation \ref{sdde}.

Furthermore, $\tau=\lim_{N} \tau_N$, is the explosion time. From the uniqueness part we know that $(X,\tau_N)$ and $(X,\tau_{N-1})$ are identical before $\tau_{N-1}$. This implies that we can extend the unique local solution to the unique maximal local solution. The existence and uniqueness is inherited.


\section{Conclusion}

In this essay, we studied stochastic differential equations with infinite delay and L\'evy noise. We proved the global results under linear growth condition, and local result under polynomial nonlinear growth condition.

This essay mainly referred from \cite{WEI2007516} and the proof in \cite[Theorem 8.3]{gall2016brownian}.

\section*{Appendix}

The following theorem is usually abbreviated as BDG inequality, \cite[Theorem 5.16]{gall2016brownian}
\begin{theorem}
[Burkholder-Davis-Gundy Inequality]\label{bdg}
For every real $p>0$, there exist two constants $c_p, C_p>0$ depending only on $p$ such that, for every continuous local martingale $M$ with $M_0=0$, and every stopping time $T$,
\begin{equation*}
c_p\e \big[\iprod{M,M}_{T}^{p/2} \big]\le \e\big[\sup_{s\le T}|M_s|^p\big]\le C_p \e\big[\iprod{M,M}_T^{p/2}\big].
\end{equation*}
\end{theorem}

% The following inequality \cite[Theorem 4.4.23]{applebaum_2009} generalize BDG inequality to the stochastic integral with respect to the Poisson random measure.

% \begin{theorem}
%     [Kunita's First Inequality] Assume that 
% \begin{equation*}
%     I(t)=\int_0^t \int_E H(s,x) \tilde{N}(\dd x\dd s).
% \end{equation*}
% Then for any $p\ge 2$, there exists $D_p>0$ such that
% \begin{equation*}
% \e\big[\sup_{s\le t} |I(s)|^p\big]\le D_p \int_0^t \int_E |H(s,x)|\nu(\dd x)\dd s 
% \end{equation*}
% \end{theorem}

 The following inequality generalize the BDG inequality to the càdlàg processes

\begin{theorem}
    [Bukholder-Davis-Gundy, Generalization] \label{bdgg}
    For every real $p\ge 2$, there exist two constants $c_p, C_p>0$ depending only on $p$ such that, for every càdlàg martingale $M$ with $M_0=0$, and every stopping time $T$,
    \begin{equation*}
        c_p\e \big[[M,M]_{T}^{p/2} \big]\le \e\big[\sup_{s\le T}|M_s|^p\big]\le C_p \e\big[[M,M]_T^{p/2}\big].
        \end{equation*}
\end{theorem}

\begin{theorem}
[Uniform Limit Preserve Càdlàg Property] \label{ucd}Let $f_n:[0,T]\to \re$ be a uniformly convergent sequence with limit $f$. Assume that $f_n$ are all càdlàg, then $f$ is càdlàg.
\end{theorem}

\begin{proof}
The proof of this result is purely a mathematical analysis stuff.

Let $x$ be the discontinuous point of $f$. For $\varepsilon>0$, exists $N>0$ such that for every $m,n>N$ we have
\begin{equation*}
    |f_m(t)-f_n(t)|\le \varepsilon/3, \quad \forall t\in [0,T].
\end{equation*}
For every $m, n$, we can choose $t_0$ near $x$ such that
\begin{equation*}
    |f_m(t_0)-f_m(x-)|\wedge |f_n(t_0)-f_n(x-)|\le \varepsilon/3.
\end{equation*}
Hence,
\begin{equation*} 
    |f_n(x-)-f_m(x-)|\le \varepsilon,\quad \forall m,n \ge N.
\end{equation*}
This implies that $f_n(x-)$ is Cauchy, hence limits in $\re$. 

Now consider
\begin{equation*}
    |f(t)-\lim_n f_n(x-)|\le |f(t)-f_n(t)|+|f_n(t)-f_n(x-)|+|f_n(x-)-\lim_n f_n(x-)|.
\end{equation*}
For fixed $\varepsilon>0$, the first and third term are less than $\varepsilon/3$ when $n>N$. Fix $n>N$, the second term is less than $\varepsilon/3$ when $x-\delta<t<x$. This implies that $\lim_n f_n(x-)=f(x-)$. 

For the second part, basically we have
\begin{equation*}
    \lim_n f_n(x+)=f(x+).
\end{equation*}
On the other hand,
\begin{equation*}
    f(x)=\lim f_n(x)= \lim f_n(x+)=f(x+).
\end{equation*}
So $f$ is càdlàg. We are done. 
\end{proof}

\bibliography{mart.bib}

\end{document}