\documentclass[8pt,onesided]{article}

% Packages.

\usepackage[usenames,dvipsnames]{xcolor}
\usepackage{amsmath}
\usepackage{mathtools}
\usepackage{amssymb}
\usepackage{setspace}
\usepackage{graphicx}
\usepackage[colorlinks=true,linktoc=all,linkcolor=blue,urlcolor=blue,citecolor=red]{hyperref}
\usepackage{amscd,amsthm,curve2e,graphicx,manfnt,mathdots,mathrsfs}
\usepackage{pgfplots}
\usepackage{graphics}
\usepackage{caption}
\usepackage{geometry}
\usepackage{fancyhdr}
\usepackage{titlesec}
\usepackage{newtxtext}
\usepackage{newtxmath}
\usepackage[ruled,linesnumbered]{algorithm2e}
\usepackage{float}
\usepackage{paralist}   
\usepackage{pstricks-add}
\usepackage{tikz-cd}
\usepackage{pst-pdf}
\usepackage{extarrows}

\usepackage{physics}
\globalcolorstrue

\titleformat{\section}{\color{BlueViolet}\normalfont\large\bf}{\color{BlueViolet}\S\thesection}{1em}{}
\titleformat{\subsection}{\color{BlueViolet}\normalfont\bf}{\color{BlueViolet}\thesubsection}{1em}{}

\geometry{left=2.0cm, right=2.0cm, top=2.5cm, bottom=2.5cm}

\fancypagestyle{style}{
    \chead{Martingale Theory and Stochastic Integration Homework 1 - Yamada Watanabe Theorem}
    \renewcommand{\headrulewidth}{.5pt}  
    \lhead{}
    \rhead{}
    \fancyfoot[L]{\color{gray}Beining Wu - Fall 2021}
    \fancyfoot[R]{\thepage}
    \fancyfoot[C]{}
    }
\pagestyle{style}


% User Commands.

\newcommand{\rnd}{\partial}
\newcommand{\dif}{\mathrm{d}\.}

\newcommand{\dq}[2]{\frac{\mathrm{d} #1 }{\mathrm{d} #2}}
\newcommand{\hdq}[3]{\frac{\mathrm{d}^{#3} #1 }{\mathrm{d} #2}^{#3}}
\newcommand{\pdq}[2]{\frac{\partial #1}{\partial #2}}
\newcommand{\hpdq}[3]{\frac{\partial^{#3} #1}{\partial #2 ^{#3}}}

\newcommand{\iprod}[1]{\left\langle #1\right\rangle}
\newcommand{\nm}[1]{\left\lVert#1\right\rVert}

\newcommand{\ex}[1]{\mathbb{E}\left[#1\right]}

\newcommand{\re}{\mathbb{R}}
\newcommand{\ra}{\mathbb{Q}}
\newcommand{\zah}{\mathbb{Z}}
\newcommand{\neu}{\mathbb{N}}
\newcommand{\s}{\mathbb{S}}
\newcommand{\p}{\mathbb{P}}
\newcommand{\f}{\mathscr{F}}
\newcommand{\B}{\mathscr{B}}
\newcommand{\w}{\mathbb{W}}
\newcommand{\q}{\mathbb{Q}}
\newcommand{\e}{\mathbb{E}}

\newcommand{\mc}{\color{BlueViolet}}
\renewcommand{\qedsymbol}{\mc $\blacksquare$}

\renewcommand{\ge}{\geqslant}
\renewcommand{\le}{\leqslant}

\definecolor{MyRed}{RGB}{117,35,31}
\newcommand{\mr}{\color{MyRed}}
\usepackage[colorlinks=true,linktoc=all,linkcolor=MyRed,urlcolor=blue,citecolor=MyRed]{hyperref}

% Theorems, etc. Environment Settings.

\theoremstyle{definition}
\newtheorem{problem}{\mc Problem}
\newtheorem*{proposition}{\mc Proposition}
\newtheorem{definition}{\mc Definition}
\newtheorem{theorem}{\mc Theorem}
\newtheorem*{example}{\mc Example}
\newtheorem*{remark}{\mc Remark}
\newtheorem{lemma}{\mc Lemma}
\newtheorem*{ass}{\mc Assumption}
\newtheorem{coro}{\mc Corollary}


\newenvironment{myenume}{\begin{enumerate}[(i).]\setlength{\itemsep}{0pt}\setlength{\itemindent}{1em}}{\end{enumerate}}

\bibliographystyle{plain}



\begin{document}

{
\title{\mc  Brief Review of Yamada-Watanabe Theorem\vspace*{.5em}\\  \large Second Homework of Martingale Theory of Stochastic Integration }
\author{Beining Wu\footnote{mail: \texttt{andrewwu@mail.ustc.edu.cn}, Student ID: PB19151833}}
\maketitle
}


\section{Introduction, Problem Settings}

\subsection{Notations and Preliminaries}

We begin with several notations that would be frequently used in this essay. 

\paragraph{Filtered Space} Our effort is generally based on $(\Omega, \f,(\f_t)_{t\ge 0}, \p)$ . Here $\f_t$ is an increasing $\sigma$-algebra, while $\p$ is the probability measure defined on the measurable space $(\Omega,\f)$. Often we would pay attention to the complete filtration, where all the null sets are measurable. Definition of measurability and $\sigma$-algebra can be found in the classical text \cite{durrett2019probability}.

\paragraph{Wiener Space} Let $W:=C(\re^+,\re^d)$ be the space of all continuous functions on $\re^+$. The measurable structure on $W$ is defined by with the canonical process $(X(w))_t:=w(t)$. Define the filtration as $\mathscr{X}_t:=\sigma \iprod{ X_s, s\le t}$, and the $\sigma$-algebra as $\mathscr{B}(W):=\sigma \iprod{\mathscr{X}_t,t\ge 0}$. One could check that this $\sigma$-algebra coincides with the Borel $\sigma$-algebra on $W$, where the topology is induced form the metric
\begin{equation*}
    d(x,y)=\sum_{k=1}^{\infty} \frac{1}{2^k} \frac{\sup_{[0,k]} |x(t)-y(t)|}{1+\sup_{[0,k]} |x(t)-y(t)|}.
\end{equation*}

\paragraph{Process} A stochastic process $(X_t)_{t\ge 0}$ is an indexed collection of  random variables, or function-valued random elements. A process is \textbf{adapted} if, for every $t\ge 0$ we have $X_t$ is $\f_t$ measurable. It's \textbf{progressive} if, for every $t\ge 0$, the mapping $X: (\omega,t)\mapsto X_t(\omega)$ is $ \f_t\otimes  \mathscr{B}([0,t])$. Moreover, we can define the \textbf{progressive $\sigma$-algebra} $\mathscr{P}$ as the sets of all the sets $A\in\f \otimes \mathscr{B}(\re^+)$ such that the process $1_{A}(\omega,t)$ is progressive. One can check that a process if progressive iff it's $\mathscr{P}$ measurable.

\paragraph{Brownian Motion and Wiener Measure} A $\f_t$-Brownian motion $(B_t)_{t\ge 0}$ is a $\f_t$ adapted process with independent increments and continuous sample paths. Moreover
\begin{equation*}
    B_t-B_s\sim  N(0,t-s).
\end{equation*}
For the construction and properties of Brownian motion, one can refer to \cite{gall2016brownian} and \cite{rogers2000diffusions1}. The wiener measure is the law of Brownian motion on $W$, i.e., the push-forward probability measure on $W$.

\subsection{What is Stochastic Differential Equations?}

Cited from \cite{gall2016brownian}, the theory of stochastic differential equations aims to provide a model for a ordinary differential equation with a random perturbation. An simple ordinary equation can be formulated as
\begin{equation*}
    y'(t)=b(y(t)),
\end{equation*}
or in the differential form,
\begin{equation*}
    \dd y_t=b(y_t) \dd t.
\end{equation*}
Now we would involve the random noise term. Clearly a Brownian motion would be an ideal noise for its properties of independent increments. The model can be unrigorously written as

\begin{equation*}
    \dd X_t=b(X_t) \dd t+\sigma \dd B_t,
\end{equation*}
where $\sigma$ is a fixed constant.

Our model would be a bit more complex than the aforementioned one. 

\begin{definition}
[Stochastic Differential Equations] Let $ \sigma: \re^+ \times \re^n \to  L(\re^d;\re^n)$, which is the matrix valued mapping, and $b:\re^+ \times \re^n \to \re^n$ be the locally bounded measurable functions. We say that the solution of the stochastic differential equation
\begin{equation} 
    \dd X_t = \sigma(t,X_t) \dd B_t + b(t,X_t) \dd t
\end{equation}
consists of
\begin{itemize}
    \item Filtered space $(\Omega, \f,(\f_t)_{t\ge 0}, \p)$.
    \item A $d$-dimensional $(\f_t)$-Brownian motion $(B_t)_{t\ge0}$.
    \item An $n$-dimensional adapted process $X_t$, with continuous sample path, such that
\begin{equation*}
    \label{sde}
    X_t=X_0+\int_0^t \sigma(s,X_s)\dd B_s+\int_0^t b(s,X_s) \dd s,
\end{equation*}
or written in coordinate form
\begin{equation*}
    X_t^i=X_0^i+\int_0^t \sum_{j=1}^d \sigma(t,X_s)_{ij}\dd B_s^j+\int_0^t b(s,X_s)_i \dd s.
\end{equation*}
\end{itemize}
\end{definition}

\begin{remark}
    [Technical Notations] Here are some assumptions that would be necessary for the definitions to be rational. First is measurability assumptions. Recall that in the definition of the stochastic integration, the integrand processes should be progressive and consequently the integral processes are adapted. The following assumption if from \cite{rogers2000diffusions2}, wherein the settings are more complex.
    \begin{ass}
        [Progressive Functional] The mapping $\sigma$ and $b$ are the progressive functional, which means, the process $\sigma(t,X_t)$ and $b(t,X_t)$ are progressive if $X_t$ is adapted.
    \end{ass}
    Another assumption comes from the rationality of stochastic integration. Note that we have required $\sigma$ and $b$ to be locally bounded. This could be replaced with a more general assumption. 
    \begin{ass}
    [Finiteness] \label{fin} The mapping $\sigma$ and $b$ should follow
    \begin{equation*}
        \forall t\ge 0 \qquad\int_0^t \nm{\sigma(s,X_s)}_{F}^2 + |b(s,X_s)| ds<\infty\quad \text{a.s.}
    \end{equation*}
    \end{ass}
    But as \cite{rogers2000diffusions2} noted, this finiteness requirement is nothing to worry much about, because we can use the stopping time trick to restrict our attention before explosion.
\end{remark}

\subsection{What We Want to Study?}

From \cite{gall2016brownian} we have known that, if we assume $\sigma$ and $b$ to be Lipschitz continuous, then the pathwise uniqueness easily follows, and we can construct the solution with Picard approximation series. However there are plenty of cases where Lipschitz continuity fails but pathwise uniqueness still holds. In Yamada and Watanabe's classical paper \cite{yamada1971}, they proved that the pathwise uniqueness implies the weak uniqueness, and moreover, there exists a strong solution. In our essay, we will go over the definition of the uniqueness and existence of \ref{sde}. Then we will go over the proof of the classical Yamada-Watanabe theorem.

\section{Definition of Existence and Uniqueness}

In this section, we will quickly go over the definition of uniqueness and existence in different senses. Here and after, we will use the term \textbf{set-up} $(\Omega, \f ,(\f_t)_{t\ge 0},\p, \xi, B)$, where $\xi$ is a $\f_0$ measurable random variable and $B=(B_t)_{t\ge 0}$ is the $\f_t$ Brownian motion.

\subsection{Weak Scenario}

We begin with the weak scenario, where all the existence and uniqueness are defined in the sense of distribution.

\begin{definition} 
[Weak Solution] We say that the SDE
\begin{equation}
    \label{wsde}
    X_t=X_0+\int_0^t b(s,X_s) \dd s+\int_0^t \sigma(s,X_s) \dd B_s
\end{equation}
has a weak solution with initial distribution $\mu$, if there exists
\begin{itemize}
    \item a $(\Omega, \f ,(\f_t)_{t\ge 0},\p)$ that satisfy the usual conditions.
    \item a $\f_t$-Brownian motion $B_t$ 
    \item a semimartingale $X_t$ such that equation \ref{wsde} and assumption \ref{fin} hold, and $X_0$ has law $\mu$.
\end{itemize}
\end{definition}

\begin{definition}
[Weak Uniqueness] We say that the solution (weak in this sense) of the equation \ref{wsde} is weakly unique, or unique in distribution, if whenever $X_t$ and $X'_t$ are both the weak solution of \ref{wsde} with same initial distribution, then $X_t$ and $X'_t$ are identical in distribution.
\end{definition}

\subsection{Strong Scenario}

\begin{definition}
[Pathwise Uniqueness] We shall say that the \textbf{pathwise uniqueness} holds for equation $\ref{sde}$, if for fixed setup $(\Omega, \f ,(\f_t)_{t\ge 0},\p, \xi, B)$ and any two continuous semimartingales $X$ and $X'$ such that
\begin{align*}
X_t=\xi+\int_0^t b(s,X_s)\dd s+\int_)^t \sigma(s, X_s)\dd B_s,\\
X_t'=\xi+\int_0^t b(s,X_s')\dd s+\int_)^t \sigma(s, X_s')\dd B_s,\\
\end{align*}
then we have
\begin{equation*}
    \text{a.s.}\quad  X_t=X_t', \forall t\ge 0
\end{equation*}
\end{definition}


\section{Main Result: Yamada-Watanabe Theorem}

\section{From the Pathwise Uniqueness to Weak Uniqueness}

\section{Existence of Strong Solution}




\bibliography{mart.bib}

\end{document}